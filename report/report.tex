\documentclass{article}
\usepackage{amsmath}
\usepackage{breqn}
\usepackage{color}
\usepackage{hyperref}
\usepackage{datetime}
\usepackage{graphicx}

\definecolor{heavyblue}{cmyk}{1,1,0,0.25}

\newcommand{\thetitle}{SEME 2016: OptionWay Project Report}
\newcommand{\theauthors}{Malcolm Roberts, Athmane Bakhta, Matteo Aletti, Boris Nectoux}

\title{\thetitle{}}
\author{\theauthors{}}

\hypersetup{
  pdftitle={\thetitle},
  pdfpagemode=UseOutlines,
  citebordercolor=0 0 1,
  colorlinks=true,
  allcolors=heavyblue,
  breaklinks=true,
  pdfauthor={\theauthors},
  pdfpagetransition=Dissolve,
  bookmarks=true
}

% Specify ISO date format:
\yyyymmdddate
\renewcommand{\dateseparator}{-}

\def\abs#1{{\left|#1\right|}}
\def\(#1\){{\left(#1\right)}}
\def\nobr#1{\hiderel{#1}}

\begin{document}

\maketitle

\begin{abstract}
  asdfasdfasdf
\end{abstract}

\section{Introduction}

Travellers wishing to purchase a flight for a specific day to a
particular destination are faced with a problem: how can they find the
cheapest flight which matches their criteria?  Since the advent of
online travel agents, it is easy to find the cheapest price available
that day, and multiple services are available.  The travel agent
offers a variety of prices from multiple airlines, and the consumer is
free to choose whether or not they wish to buy.  At this point one
would think that the problem is solved, but one must remember that
there is another player in this particular economic game: the airline,
which seeks to sell seats on planes in order to maximize its own
profit.

In general, there are several airlines which compete on a particular
route, such as Paris--New York, and the airlines will vary their prices
according to their own individual strategies (strategies which, we add
parenthetically, produce seemingly bizarre prices).  The airlines
compete with each other, but they can also increase their profit by
adopting a strategy which increases the likelihood that an individual
consumer will purchase a ticket.  In addition, there are circumstances
where a consumer has very low price sensitivity, so it might be good
to occasionally offer higher prices in case such a client is looking
for a flight at the same moment when the high prices are in effect.

The effect of this on the consumer is that there is a significant
amount of variance in the price of the cheapest flight available.  If
the Paris--New York flight is \$1000 today, it might be \$900 next
week, or perhaps \$1100.  Thus the optimal strategy for the consumer
is to check the cost each day to determine the range of prices
available and then to use that information in order to try and get a
better price.  Each day the traveller will visit various websites,
collect data, and, when they think that they have enough information,
start looking for a flight.  While there may certainly be some
travellers who find this sort of activity a great way to spend a
Friday night, the vast majority will probably find the use of their
time sub-optimal for what may end up being a fairly minimal savings in
cost.

The online travel agent OptionWay autmoates this process; the
traveller specifies their destination and travel dates, the price that
they are willing to pay for the trip, and the length of time that they
want OptionWay to look for the ticket.  If their criteria are
satisfied during this time, OptionWay automatically buys the ticket
when the specified price is available.  Using this method, the
traveller is saved the drudgery of manually searching for flights on a
daily basis.

However, the demand price may or may not be likely to appear in the
specified search period.  For example, if the traveller wants to fly
Paris--New York for \$400, they are likley to never find a ticket at
such a price.  OptionWay would like to provide an estimate for the
probability of the chance of success of their demand.

\section{Data and Analysis}

OptionWay provided us with sample data for flights to various
destinations.  The data was composed of all the flight prices for a
particular route for travel on a variety of dates, with flight prices
available from 130 days before the flight to the day of.  Data was not
necessarily available for each day.  There were 12 routes available.
Data was given in the form of csv files, which ranged in size from 13M
to 2.3G.



\section{Conclusion}

\end{document}
